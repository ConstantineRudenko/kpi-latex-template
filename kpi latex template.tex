% allows using @ in macro
\makeatletter

% pt in MS Word is not equal to pt in LaTeX
% pt in MS Word is equal to bp in LaTeX 
% https://tex.stackexchange.com/questions/34024/setting-a-document-in-ms-word-12pt-12bp
% this will globally replace size of pt to match MS Word
\p@=1bp

% disallows using @ in macro
\makeatother

% article does not support 14pt
\documentclass[14pt]{extarticle}

%set margins and page format
%call geometry first so everything else uses set margins
\usepackage[a4paper, portrait, includefoot, left=3cm, right=2cm, top=2cm, bottom=2cm, footskip=14pt]{geometry}

\usepackage{fancyhdr}
\renewcommand{\headrulewidth}{0pt}
\renewcommand{\footrulewidth}{0pt}
\pagestyle{fancy}
\fancyhf{}
\rfoot{\thepage}

%set line spacing
\renewcommand{\baselinestretch}{1.5} 

\setlength\parindent{0pt}

% allows using @ in macro
\makeatletter

% change bibliography item numbering style
\renewcommand\@biblabel[1]{#1.}

% dotted line for table of contents (vanilla, not needed with tocloft)
%\renewcommand*\l@section{\@dottedtocline{1}{0em}{1em}}

% disallows using @ in macro
\makeatother

% ignore font substitution warnings
% it's the default font being substituted with polyglossia fonts

% first section number
%\setcounter{section}{0}

% customize title format and spacing for section, subsection, etc.
\usepackage{titlesec}
% a flag for unnumbered sections
\newcommand{\issectionnumbered}{1}
\titleformat{\section}[block]{\bfseries\filcenter}{}{0em}{\if\issectionnumbered1\thesection. \fi}{}
\titleformat{\subsection}[block]{\bfseries}{}{0em}{\thesubsection. }{}
\titlespacing{\section}{0pt}{0pt}{0pt}
\titlespacing{\subsection}{0pt}{0pt}{0pt}

% custom command for unnumbered sections
\newcommand{\unnumberedsection}[1]{
\renewcommand{\issectionnumbered}{0}
\section*{#1}
\addcontentsline{toc}{section}{#1}
\renewcommand{\issectionnumbered}{1}
}

% customize table of contents
\usepackage{tocloft}
% use page style from fancyhdr for table of contents page (header/footer)
\tocloftpagestyle{fancy}
% section font in table of contents
\renewcommand\cftsecfont{\normalfont}
% section page number font in table of contents
\renewcommand\cftsecpagefont{\normalfont}
% dotted lines in table of contents
\renewcommand{\cftsecleader}{\cftdotfill{\cftdotsep}}
% dot after section number
\renewcommand{\cftsecaftersnum}{.}%
% dot after subsection number
\renewcommand{\cftsubsecaftersnum}{.}%
% center the title of table of contents
\renewcommand{\cfttoctitlefont}{\bfseries\hspace*{\fill}}
\renewcommand{\cftaftertoctitle}{\hspace*{\fill}}
% \hfill = hspace{\fill}
% use hspace*{\fill} instead
% hspace without asterisk sometimes drops spaces (like in the end of the line)
% hspace with asterisk always creates space

% don't skip vertical space before table of contents entries
\setlength\cftbeforesecskip{0pt}
\setlength\cftbeforesubsecskip{0pt}

% don't skip vertical spacebe before and after table of contents title
\setlength\cftbeforetoctitleskip{0pt}
\setlength\cftaftertoctitleskip{0pt}

\usepackage{cite}

% load polyglossia in the end to prevent it's changes being overwritten
\usepackage{polyglossia}
\setmainlanguage{ukrainian} 
\setotherlanguage{english}
\addto\captionsukrainian{\renewcommand{\contentsname}{ЗМІСТ}}

% ukrainian font for polyglossia
\newfontfamily\ukrainianfont[Script=Cyrillic,Ligatures=TeX,SizeFeatures={Size=14}]{Times New Roman}
% english font for polyglossia
\newfontfamily\englishfont[Script=Latin,Ligatures=TeX,SizeFeatures={Size=14}]{Times New Roman}

\begin{document}

\tableofcontents
\newpage

\vspace{1in}

\unnumberedsection{ВСТУП}

Текст вступу.

\section{РОЗДІЛ 1}
\subsection{Пункт 1}
Тестовий текст тестовий текст тестовий текст тестовий текст тестовий текст тестовий текст тестовий текст тестовий текст тестовий текст тестовий текст тестовий текст тестовий текст тестовий текст тестовий текст тестовий текст тестовий текст тестовий текст тестовий текст тестовий текст тестовий текст тестовий текст тестовий текст тестовий текст тестовий текст тестовий текст тестовий текст тестовий текст тестовий текст тестовий текст тестовий текст тестовий текст тестовий текст тестовий текст тестовий текст тестовий текст тестовий текст тестовий текст тестовий текст тестовий текст тестовий текст тестовий текст тестовий текст тестовий текст тестовий текст тестовий текст тестовий текст тестовий текст тестовий текст тестовий текст тестовий текст тестовий текст тестовий текст тестовий текст тестовий текст тестовий текст тестовий текст тестовий текст тестовий текст тестовий текст тестовий текст тестовий текст тестовий текст тестовий текст тестовий текст тестовий текст тестовий текст тестовий текст тестовий текст тестовий текст тестовий текст тестовий текст тестовий текст тестовий текст тестовий текст тестовий текст тестовий текст тестовий текст тестовий текст тестовий текст тестовий текст тестовий текст тестовий текст тестовий текст тестовий текст тестовий текст тестовий текст тестовий текст тестовий текст тестовий текст тестовий текст тестовий текст тестовий текст тестовий текст тестовий текст тестовий текст тестовий текст тестовий текст тестовий текст тестовий текст тестовий текст тестовий текст тестовий текст тестовий текст тестовий текст тестовий текст тестовий текст тестовий текст тестовий текст тестовий текст тестовий текст тестовий текст тестовий текст тестовий текст тестовий текст тестовий текст тестовий текст тестовий текст тестовий текст тестовий текст тестовий текст тестовий текст тестовий текст тестовий текст тестовий текст тестовий текст тестовий текст тестовий текст.
\subsection{Пункт 2}
текст
\section{РОЗДІЛ 2}
\subsection{Пункт 1}
посилання \cite{bib1,bib3,bib4,bib5}
\subsection{Пункт 2}
текст

\begin{thebibliography}{9}
\bibitem{bib1} 
Посилання 1
\bibitem{bib2} 
Посилання 2
\bibitem{bib3} 
Посилання на
\begin{otherlanguage}{english}%
"English book"%
\end{otherlanguage}
3

\bibitem{bib4}
Посилання 4

\bibitem{bib5}
Посилання 5

\end{thebibliography}

\end{document}